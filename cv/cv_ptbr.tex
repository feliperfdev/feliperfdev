\documentclass[11pt,a4paper]{article}
\usepackage[utf8]{inputenc}
\usepackage[T1]{fontenc}
\usepackage[margin=1cm]{geometry}
\usepackage{hyperref}
\usepackage{enumitem}
\usepackage{titlesec}

\usepackage{helvet}
\renewcommand{\familydefault}{\sfdefault}

% Configurações de hyperlinks
\hypersetup{
    colorlinks=true,
    linkcolor=blue,
    urlcolor=blue,
    citecolor=blue
}

% Configurações de seções
\titleformat{\section}{\large\bfseries}{}{}{}[\titlerule]
\titleformat{\subsection}{\normalsize\bfseries}{}{}{}
\titlespacing{\section}{0pt}{12pt}{6pt}
\titlespacing{\subsection}{0pt}{8pt}{4pt}

% Remove indentação de parágrafos
\setlength{\parindent}{0pt}

\begin{document}

% Nome
{\LARGE\bfseries Felipe Azevedo Ribeiro}

\vspace{0.3cm}

% Contatos
\href{mailto:feliper.dev@gmail.com}{feliper.dev@gmail.com} | 
\href{https://github.com/feliperfdev}{github.com/feliperfdev} | 
\href{https://www.linkedin.com/in/feliperdev}{linkedin.com/in/feliperdev} | 
\href{https://www.feliperdev.link}{feliperdev.link} | 
\href{tel:+5571996446771}{(71) 996446771}

\section*{Resumo}

Desenvolvedor Mobile Pleno especializado em Flutter, com expertise em desenvolvimento de aplicações nativas para Android e iOS. Focado em entrega contínua com código limpo, testável e de fácil manutenção, colaboração técnica e code reviews. Garanto excelência em UI/UX com fidelidade pixel-perfect ao design e otimização de performance. Comprometido com agilidade, proatividade e melhoria contínua.

\section*{Educação}

\textbf{Engenharia de Computação} — Universidade SENAI Cimatec — Salvador/BA — 2020–2025

\section*{Habilidades}

\begin{itemize}[leftmargin=*, itemsep=1pt]
    \item \textbf{Mobile:} Flutter, Dart, BLoC, Provider, Riverpod, MobX, Drift ORM, Secure Storage, Geolocalização, Flutter Web, interoperabilidade Android/Kotlin, Push Notifications (FCM), Offline-first, Caching
    \item \textbf{Backend:} Node.js, TypeScript, Express.js, NestJS
    \item \textbf{Cloud/DevOps:} AWS Lambda, API Gateway, DynamoDB, Cognito, Docker, CI/CD, GitHub Actions, Fastlane
    \item \textbf{Dados:} PostgreSQL, Supabase, Firebase, MySQL, SQLite
    \item \textbf{Arquitetura/Qualidade:} Clean Architecture, Hexagonal Architecture, MVVM, DDD, TDD, REST APIs, gRPC, Protocol Buffers, Unit/Widget Tests, Sentry, Mixpanel
    \item \textbf{Design:} Figma, Rive, Lottie
    \item \textbf{Idiomas:} Português (Nativo), Inglês (B2)
\end{itemize}

\section*{Experiência profissional}

\subsection*{Engenheiro de Software | F.R Dev | 2024–2025}

\begin{itemize}[leftmargin=*, itemsep=1pt]
    \item Projetou e desenvolveu Micro-SaaS web e mobile com Flutter e backend serverless (AWS/Supabase) aplicando DDD, arquitetura hexagonal e estratégias \textit{offline-first} com sincronização, \textit{sync-queue} e caching.
    \item Construiu APIs REST com Node.js e Express.js priorizando segurança e escalabilidade.
    \item Implantou CI/CD, testes unitários/widgets e Docker, elevando a confiabilidade e a velocidade de releases.
    \item Assegurou fidelidade pixel-perfect ao design e otimizou performance, monitorando tempo de resposta e consumo de memória.
\end{itemize}

\subsection*{Desenvolvedor Mobile | Petize | 2023–2025}

\begin{itemize}[leftmargin=*, itemsep=1pt]
    \item Aplicou Arquitetura Hexagonal com TDD, alta cobertura em módulos core e redução de bugs em produção.
    \item Implementação o padrão Bloc para gerenciamento de estado, garantindo a separação de responsabilidades entre a interface e a regra de negócio; Otimizou animações Lottie, garantindo 60fps consistentes.
    \item Automatizou o pipeline de releases (Fastlane + CI/CD) com testes automatizados.
    \item Liderou integrações de REST APIs e push notifications (Firebase Cloud Messaging).
\end{itemize}

\subsection*{Desenvolvedor Mobile | Ommed | 2021}

\begin{itemize}[leftmargin=*, itemsep=1pt]
    \item Implementou Clean Architecture + TDD em features críticas.
    \item Estabeleceu comunicação entre microserviços via gRPC/Protocol Buffer.
    \item Gerenciou estado complexo com MobX e fluxos assíncronos reativos.
\end{itemize}

\section*{Projetos}
\begin{itemize}[leftmargin=*, itemsep=1pt]
    \item Aibum — Geração de imagens para colorir. Flutter Web; APIs com Supabase e Express.js;
    \item KeepLinked — Organização de links com pastas/tags/privacidade. Flutter; APIs com Supabase/Express.js.
    \item Wander Level — App mobile \textit{offline-first} com sync-queue. Flutter; Drift ORM; Supabase;
    \item FinanceTracker — App mobile para finanças pessoais. Flutter; backend AWS (DynamoDB, Lambda, API Gateway, Cognito).
\end{itemize}

\end{document}